\documentclass[UTF8]{ctexart}
\usepackage{mathtools,wallpaper}
\usepackage{lmodern}
\usepackage{t1enc}
\usepackage{pagecolor}
\usepackage{booktabs}
\usepackage{amsmath}
\usepackage{amsthm}
\renewcommand{\(}{\left(}
\renewcommand{\)}{\right)}
\begin{document}
\title{算法设计 HW01}  
\author{吴硕 522030910094}
\maketitle

\paragraph{Q1} 
\begin{proof}

根据分支算法的思想,整个$n$可以分为$\log_{b}{n}$层,最后一层是$a^{\log_{b}{n}}$个元素。
因此,总的时间开销为:
$$
\begin{aligned}
	sum&=\sum_{i=0}^{\log_{b}{n}} {a^{i}\(\frac{n}{b^{i}} \)^{d}\(\log_{}{\frac{n}{b^{i}}}\)^{w}}  \\
    &=n^{d}\sum_{i=0}^{\log_{b}{n}} {\(\frac{a^{i}}{{b^{id}}} \)}\(\log_{}{\frac{n}{b^{i}}}\)^{w} 
\end{aligned}
$$
又不妨设$n=b^{k}$,可得:
$$
\begin{aligned}
    sum&=n^{d}\sum_{i=0}^{\log_{b}{n}} {\(\frac{a^{i}}{{b^{id}}} \)}\(\log_{}{{b^{k-i}}}\)^{w} \\
    &=n^{d}\(\log_{}{b}\)^{w}\sum_{i=0}^{\log_{b}{n}} {\(\frac{a}{{b^{d}}} \)^{i}}\(k-i\)^{w}  \\
\end{aligned}
$$
我们取$c=1/b^{d}$

1.若$a<b^{d}$
$$
\begin{aligned}
    LHS &=n^{d}\(\log_{}{b}\)^{w}\(c^{0}k^{w}+c^{1}\(k-1\)^{w}+...+c^{k}0^{w}\)\\
 &<n^{d}\(\log_{}{b}\)^{w}k^{w}\(c^{0}+c^{1}+...+c^{k}\)\\                 
 &=n^{d}\(\log_{}{b}\)^{w}k^{w}\frac{1-c^{k+1}}{1-c}\\
&<n^{d}\(\log_{}{n}\)^{w}
\end{aligned}
$$

所以$a<b^{d}$时,$sum=O\(n^{d}\(\log_{}{n}\)^{w}\)$
    

2.若$a=b^{d}$
$$
\begin{aligned}
    LHS&=n^{d}\(\log_{}{b}\)^{w}\(k^{w}+\(k-1\)^{w}+...+0^{w}\) \\
&<n^{d}\(\log_{}{b}\)^{w}k^{w}\\
&=n^{d}\(\log_{}{n}\)^{w}\\
&=n^{d}\(\log_{}{n}\)^{w+1}
\end{aligned}
$$


所以$a=b^{d}$时,$sum=O\(n^{d}\(\log_{}{n}\)^{w+1}\)$

3.若$a>b^{d}$
$$
\begin{aligned}
LHS&=n^{d}\(\log_{}{b}\)^{w}\(c^{0}k^{w}+c^{1}\(k-1\)^{w}+...+c^{k}0^{w}\)\\
&=n^{d}\(\log_{}{b}\)^{w}c^{k}\(\frac{1}{c^{k}}k^{w}+\frac{1}{c^{k-1}}\(k-1\)^{w}+...+\frac{1}{c^{0}}0^{w}\)\\
&=n^{\log_{b}{a}}\(\log_{}{b}\)^{w}\(\frac{1}{c^{k}}k^{w}+\frac{1}{c^{k-1}}\(k-1\)^{w}+...+\frac{1}{c^{0}}0^{w}\)
\end{aligned}
$$

又由d'Alembert判别法得,括号中的项是一个收敛的级数,所以
$$LHS<C*n^{\log_{b}{a}}$$

所以$a>b^{d}$时,$sum=O\(n^{\log_{b}{a}}\)$

\end{proof}
\paragraph{Q2} 

(a) 首先,先随机选择一个数,然后将数组分为两部分,一部分比它小,一部分比它大。
从期望的意义下,我们将n分为至少1/3和2/3的两部分,因此根据主定理
$$T\(n\)<2T\(\frac{n}{\frac{2}{3}}\)+O\(\frac{2}{3}n\)$$
因此可得$T\(n\)=O\(n^{\log_{\frac{3}{2}}{2}}\)$


(b) 对于任意的i,j,我们考虑$x_i,x_j$两个数被比较到的概率,为$\frac{2}{j-i+1}$
因此,所有比较次数的期望为
$$
\begin{aligned}
    E=&\sum_{i=1}^{n-1}\frac{2}{i+1}\(n-i\)\\
&=\frac{2}{2}\(n-1\)+\frac{2}{3}\(n-2\)\frac{2}{4}\(n-3\)+...+\frac{2}{n}\(n-\(n-1\)\)\\
&=2\(\sum_{i=1}^{n-1}{\frac{n+1}{i+1}-n+1}\)\\
&=2\(n+1\)\sum_{i=1}^{n-1}{\frac{1}{i+1}}-2n+2\\
&<2\(n+1\)\int_{1}^{n}\frac{1}{n}\\
&=2\(n+1\)\ln{n}\\
&=O\(n\ln{n}\)
\end{aligned}$$

Q.E.d


\paragraph{Q3}
首先不妨假设$n=2^k$

我们考虑分治的做法,首先递归地将n分为等大的两部分,最终分到每组只有一个元素。
然后在每次向上合并时,每次比较两组的值,记录下比较的结果,并将大的值向上传递,
最终合并到最上层时即可得到最大的值。此时由于每层合并都要比较$\frac{n}{2}$次,
所以找到最大值所需要的比较次数是n-1次。

接着,考虑找到第二大的值,我们根据之前找到的最大值,记为A,以及储存的比较结果,取最后一次和A比较的
值,记为B,接着逐层向下将B和每个和A比较过的值进行比较,每次取大的结果作为新的B,最终得到第二大的值。
由于每层比较都要比较一次,共需要比较$\log{n}-1$次,因此总的比较次数为$n+\log{n}-2$次

Q.E.D

\paragraph{Q4}
(a) 考虑一个$n*1$的矩阵,我们每次取其中第$\frac{n}{2}$个数据然后比较矩阵左右两端和这个数据的大小。

如果两个数中有一个数比它小,记为A,则我们取中间数到A,作为新的分治目标。同时考虑A的旁边值,记为$A_1$,如果$A_1>A$,则已经找到局部最小值,
否则,再去考虑$A_2,A_3$,重复以上过程,如果重复至中间值,且可得序列中间值$>A>A_1>A_2....$,因此可得
最后的$A_\frac{n-1}{2}$即为局部最小值。因此可得,在新的分治目标中一定存在局部最小值。

如果两个数都比中间数要大,则先考虑中间值是是否为局部最小值,如果是,则找到,如果不是,则存在$B<$中间值,我们将中间值到B方向的端点
作为新的分治目标,并同理A的过程,可得新的分治目标中也一定存在局部最小值。

因此综上,每一次的对半分支都可以确保新的分治目标中有局部最小值,且一定能够被找到,因此该算法正确。

又每次对半分治,该算法复杂度为$O(\log{n})$

(b) 首先考虑最外侧两列,两行,中间行,中间列共$6n$个数据,找到其中的最小值,记为$A$,然后比较$A$的上下左右四个数,
如果$A$是局部最小值,则找到,否则,如果$A$的上下左右四个数中有一个比$A$小,则取该数所在的$\frac{1}{4}$矩阵(包括边界)作为新的分治目标。

现在考虑这个新的矩阵中的最小值,如果该最小值在边界上,则与$A$以及边界的取法矛盾,如果在矩阵里面,则局部最小值在新的分治目标中。
因此该算法可以确保新的分治目标中有局部最小值,且一定能够被找到,因此该算法正确。

考虑时间复杂度,第一次比较需要$6n$,接下去的每次比较的和$<2n$,因此总的时间复杂度为$O(n)$。

(c) 若(b)中的情况在有任何一遍达到1时,改为(a)中的方法,不妨设$m<n$,则可得时间复杂度通解为$T=O(n\log{m}+\log{n})$

\paragraph{Q5}

(a) 如果$x$,$y$,$z$并不同奇同偶,不妨设$x$,$y$为奇数,$z$为偶数,则$y-x$为偶数,$z-x$为奇数,则$x$,$y$,$z$一定无法形成三元组,因此
由逆否命题得,如果$x$,$y$,$z$形成三元组,则它们一定同奇同偶。

(b) $\{2,1,4,3\}$,

$1-2!=4-1,1-2!=3-1,4-2!=3-4,4-1!=3-4$

(c) 若$x$,$y$,$z$形成三元组,
则$\frac{y+1}{2}-\frac{x+1}{2}=\frac{y-x}{2}=\frac{z-y}{2}=\frac{z+1}{2}-\frac{y+1}{2}$
因此这三个数也构成三元组。
若这三个数构成三元组,
则$\frac{y-x}{2}=\frac{y+1}{2}-\frac{x+1}{2}=\frac{z+1}{2}-\frac{y+1}{2}=\frac{z-y}{2}$
因此$x$,$y$,$z$形成三元组。

Q.E.D

(d)对于$1-n$,我们先将它们按照奇偶分治,对于同奇偶的一组,奇数我们 就考虑它们$(+1)/2$的奇偶性,偶数我们就考虑它们$/2$的奇偶性
并不断向下分治,直到每组只有一个数,然后再逐层向上合并,每次合并时,
\end{document}
